\documentclass[a4paper, 11pt]{article}
\usepackage{longtable}
% LaTeX2e korrigieren.
\usepackage{fixltx2e}
% deutsche Spracheinstellungen
\usepackage{polyglossia}
\setmainlanguage{german}
\usepackage{diagbox}
\usepackage{fullpage}
% unverzichtbare Mathe-Befehle
\usepackage{amsmath}
% viele Mathe-Symbole
\usepackage{amssymb}
% Erweiterungen für amsmath
\usepackage{mathtools}

% Fonteinstellungen
\usepackage{fontspec}
\defaultfontfeatures{Ligatures=TeX}

\usepackage[
  math-style=ISO,    % \
  bold-style=ISO,    % |
  sans-style=italic, % | ISO-Standard folgen
  nabla=upright,     % |
  partial=upright,   % /
]{unicode-math}

\setmathfont{Latin Modern Math}
\setmathfont[range={\mathscr, \mathbfscr}]{XITS Math}
\setmathfont[range=\coloneq]{XITS Math}
\setmathfont[range=\propto]{XITS Math}
% make bar horizontal, use \hslash for slashed h
\let\hbar\relax
\DeclareMathSymbol{\hbar}{\mathord}{AMSb}{"7E}
\DeclareMathSymbol{ℏ}{\mathord}{AMSb}{"7E}

% richtige Anführungszeichen
\usepackage[autostyle]{csquotes}

% Zahlen und Einheiten
\usepackage[
  locale=DE,                   % deutsche Einstellungen
  separate-uncertainty=true,   % Immer Fehler mit \pm
  per-mode=symbol-or-fraction, % m/s im Text, sonst Brüche
]{siunitx}

% chemische Formeln
\usepackage[version=3]{mhchem}

% schöne Brüche im Text
\usepackage{xfrac}

% Floats innerhalb einer Section halten
\usepackage[section, below]{placeins}
% Captions schöner machen.
\usepackage[
  labelfont=bf,        % Tabelle x: Abbildung y: ist jetzt fett
  font=small,          % Schrift etwas kleiner als Dokument
  width=0.9\textwidth, % maximale Breite einer Caption schmaler
]{caption}
% subfigure, subtable, subref
\usepackage{subcaption}
%\usepackage{subfigure}

% Grafiken können eingebunden werden
\usepackage{graphicx}
% größere Variation von Dateinamen möglich
\usepackage{grffile}

% Standardplatzierung für Floats einstellen
\usepackage{float}
\floatplacement{figure}{htbp}
\floatplacement{table}{htbp}

% schöne Tabellen
\usepackage{booktabs}

% Seite drehen für breite Tabellen
\usepackage{pdflscape}

\usepackage{icomma}

% Mars und Venus
\usepackage{marvosym}

% Hyperlinks im Dokument
\usepackage[
  unicode,
  pdfusetitle,    % Titel, Autoren und Datum als PDF-Attribute
  pdfcreator={},  % PDF-Attribute säubern
  pdfproducer={}, % "
]{hyperref}
% erweiterte Bookmarks im PDF
\usepackage{bookmark}

% Trennung von Wörtern mit Strichen
\usepackage[shortcuts]{extdash}

\usepackage[ddmmyyyy]{datetime}
\renewcommand{\dateseparator}{.}
\usepackage[backend=biber]{biblatex}
\addbibresource../lit.bib}

\usepackage{verbatim}
\usepackage{ulem}
\usepackage{braket}
\newcommand\OverfullCenter[1]{\noindent\makebox[\linewidth]{#1}}
\begin{document}
\noindent
\centerline{\small{\textsc{Technische Universität Dortmund}}} \\
\large\textbf{Übungsblatt 09} \hfill \footnotesize\textbf{Sebastian Bange, Alexander Harnisch, Alexander Knodel} \\
\normalsize Computational Physics \hfill \today \\
Prof. Dr. Jan Kierfeld \hfill Abgabefrist: 24.06.2016\\
\noindent\makebox[\linewidth]{\rule{\textwidth}{0.4pt}}
\section*{Matrixdiagonalisierung}
In der nachfolgenden Tabelle \ref{table:1aeigenvalues} finden sich die Eigenwerte der symmetrischen Matrix, bestimmt durch die in Eigen integrierte Methode \textsc{eigenvalues()} (linke Spalte) und die über die Potenzmethode bestimmten Eigenwerte (rechte Spalte).
\begin{table}[]
\centering
\caption{Matrixeigenwerte mittels der \textsc{eigenvalues()}-Methode von Eigen (li) und über die Potenzmethode (re)}
\label{table:1aeigenvalues}
\begin{tabular}{cc}
10.699327328977576101  & 10.699327297942717863   \\
3.3594432002196952247  & 3.3594394894725887468   \\
3.0000000000000026645  & 3.0000035517627194892    \\
-2.0587705291972708821 & -2.0587715700091009374 \\
\end{tabular}
\end{table}
Es zeigt sich eine Übereinstimmung bis mindestens in die fünfte geltende Ziffer. Beide Varianten sind also sehr genau. Die Genauigkeit haben wir bei $\epsilon = 10^{-6}$ gewählt. Im Gegensatz zum Skript haben wir nicht zwei verschiedene Vektoren $\vec{\omega}$ und $\vec{v_{n-a}}$ (normierter $\omega$-Vektor, S. 149) gewählt, sondern die Iteration über eine Variable durchgeführt und diese direkt normiert. Da wir keine Imaginäranteile haben, können wir $\braket{\vec{v}_n | \uuline{A}\vec{v}_n}$ direkt über das transponierte $\omega$ ($\omega^T \uuline{A}\,\,\omega$) berechnen.

\section*{Anharmonischer Oszillator}
In dieser Aufgabe wird die numerische Lösung der stationären Schrödingergleichung auf die Diagonalisierung einer endlich-dimensionalen Matrix zurückgeführt, indem das qm. System in einem endlich-dimensionalen Unterraum des Hilbertsraumes betrachtet wird. Damit sollen die Energieeigenwerte des anharmonischen Oszillators berechnet werden.
Mit $x = \alpha \xi \left( \frac{\partial \xi}{\partial x} = \frac{1}{\alpha}\right)$ mit der Anharmonizität $\xi$  folgt:

\begin{equation}
	\begin{split}
		\frac{\partial}{\partial x} \left( \frac{\partial \psi}{\partial x} \right) &= \frac{\partial}{\partial x} \left( \frac{\partial \xi}{\partial x} \frac{\partial \psi}{\partial \xi} \right) = \frac{\partial}{\partial x} \left( \frac{1}{\alpha} \frac{\partial \psi}{\partial \xi} \right) \\
		&= \frac{1}{\alpha} \frac{\partial}{\partial \xi} \left( \frac{\partial \psi}{\partial x} \right) = \frac{1}{\alpha} \frac{\partial}{\partial \xi} \left( \frac{1}{\alpha} \frac{\partial \psi}{\partial \xi} \right) = \frac{1}{\alpha^2} \frac{\partial^2 \psi}{\partial \xi^2}
	\end{split}
	\label{eq:Ableitung umformen}
\end{equation}
Diese Ersetzung mit $x = \alpha \xi$ fügen wir nun in die Schrödinger-Gleichung für den anharmonischen Oszillator ein.
Mit $E = \beta \epsilon$ folgt
\begin{equation}
		\left( - \frac{\hbar}{2m\alpha^2} \partial_{\xi}^2 + \frac{1}{2} m\omega^2\alpha^2\xi^2 + \lambda \alpha^4 \xi^4 \right) \psi = \beta \epsilon \psi
\end{equation}
Um die Form der S-Gl aus Gleichung 4 (vgl. Zettel) zu erhalten, muss gelten:
\begin{align}
	\rightarrow \frac{\hbar^2}{2m\alpha^2\beta}&\stackrel{!}{=} 1 \label{eq:erstersummandvonHamiltonian} \\
	\frac{m\omega^2\alpha^2}{2 \beta} &\stackrel{!}{=} 1 \label{eq:zweitersummandvonHamiltonian}\\
	\tilde{\lambda} &= \frac{\alpha^4}{\beta} \lambda
\end{align}
Aus den Gleichungen \ref{eq:erstersummandvonHamiltonian} und \ref{eq:zweitersummandvonHamiltonian} entsteht ein Gleichungssystem mit 2 Gleichungen und 2 Variablen:
\begin{align}
	\ref{eq:erstersummandvonHamiltonian} \Leftrightarrow \beta &= \frac{\hbar^2}{2m\alpha^2} \qquad \beta \text{ in Gl.}\ref{eq:zweitersummandvonHamiltonian} \\
	\rightarrow \frac{\omega^2 \alpha^4 m^2}{\hbar^2} &= 1 \\
	\leftrightarrow \alpha &= \sqrt{\frac{\hbar}{m\omega}} \\
	\rightarrow \beta &= \frac{\hbar \omega}{2}
\end{align}
Damit ergibt sich nun die dimensionslose Form der S-Gl.
\begin{align}
	\left( - \partial_{\xi}^2+\xi^2+\lambda \frac{\hbar^2}{m\omega^2}\frac{2}{\hbar \omega} \xi^4 \right) \psi &= \epsilon \psi \\
	\rightarrow \tilde{\lambda} &= \lambda \frac{ 2 \hbar }{m^2 \omega^3} \\
	\left(\text{Probe}\right)[\tilde{\lambda}] &= \frac{\mathup{J}}{\mathup{m^4}} \frac{\mathup{Js}}{\mathup{kg^2} \left(\frac{1}{\mathup{s}}\right)^3} \\
	&= \mathup{J^2} \frac{\mathup{s^4}}{\mathup{m^4kg^2}} \\
	&= \left( \frac{\mathup{kg m^2}}{\mathup{s^2}}\right)^2 \frac{\mathup{s^4}}{\mathup{m^4kg^2}}\\
	&= 1
\end{align}

Nun wird der Ortsraum mittels $\xi_n = \mathup{n} \Delta \xi$ diskretisiert und $\xi$ auf dem Intervall $\xi \in [-L,L]$ betrachtet. Für $\lambda = 0$ ergeben sich folgende Eigenwerte.
\begin{table}[]
\centering
\caption{Matrixeigenwerte des Hamiltonians für $\lambda$ = 0}
\label{table:2b0eigenvalues}
\begin{tabular}{c}
0.99937460864049720843 \\
2.9968714733399832717 \\
4.9918612669583364294 \\
6.9843392427045509763 \\
8.9743006237199232089 \\
10.961740602763113017 \\
12.9466543418897988 \\
14.929036972127596172 \\
16.908883593146576629 \\
18.886189272925243188 \\
\end{tabular}
\end{table}
Die Eigenwerte liefern (mit numerischer Unsicherheit) das Resultat $\epsilon_n = 2n+1$.

Nun wird derselbe Hamiltonian in der Besetzungszahldarstellung betrachtet.
Die in Gleichung (7) auf dem Blatt gegebene Matrix soll auf Symmetrie untersucht werden. Dabei wird der Faktor 1/4 hier nicht extra aufgeführt. Für den ersten Summanden ergibt sich:

\begin{align}
& \left[m(m-1)(m-2)(m-3)\right]^{1/2} \delta_{n\,m-4} \\
= &\left[(n+4)(n+3)(n+2)(n+1)\right]^{1/2} \delta_{n+4\,m}\\
= &\left[(n+4)(n+3)(n+2)(n+1)\right]^{1/2} \delta_{m\,n+4}\\
&\text{Benenne } m\leftrightarrow n \text{ um}\\
= & \left[(m+4)(m+3)(m+2)(m+1)\right]^{1/2} \delta_{n\,m+4} \label{eq:zweiterSummandXiHoch4inBraketMatrix}
\end{align}
Gleichung \ref{eq:zweiterSummandXiHoch4inBraketMatrix} entspricht dem zweiten Summanden aus Formel 7. Der Beweis für die anderen Summanden ergeben sich analog.

Mit $\xi = (a+a^+)/\sqrt{2}$ ergibt sich

\begin{align}
	a^+ &= \frac{1}{\sqrt{2}} \left( \xi - \partial_{\xi} \right) \qquad a = \frac{1}{\sqrt{2}} \left( \xi +  \partial_{\xi} \right)\\
	a^{+^2} &= \frac{1}{2} \left( \xi^2 + \partial_{\xi}^2 - \xi \partial_{\xi} - \partial_{\xi} \xi \right) \\
	a^2 &= \frac{1}{2} \left( \xi^2 + \partial_{\xi}^2 + \xi \partial_{\xi} + \partial_{\xi} \xi \right) \\	
	a^+a &= \frac{1}{2} \left( \xi^2 + \xi \partial_{\xi} - \partial_{\xi} \xi - \partial_{\xi}^2 \right) \\
	aa^+ &= \frac{1}{2} \left( \xi^2 - \xi \partial_{\xi} + \partial_{\xi} \xi - \partial_{\xi}^2 \right) \\
	\xi^2 &= \frac{1}{2} \left( a^{+^2} + a^2 + a^+a+aa^+\right) \\
	\partial_{\xi}^2 &= \frac{1}{2} \left( a^{+^2} + a^2 - a^+a-aa^+\right)
\end{align}
Wir betrachten und erhalten
\begin{align}
	 2 \braket{n|\partial_{\xi}^2|m} &= \braket{n|a^{+^2}|m} + \braket{n|a^2|m} - \braket{n|a^+a|m}-\braket{n|aa^+|m} \\
	&= \sqrt{(m+1)(m+2)} \delta_{n\,m-2} + \sqrt{m(m-1)} \delta_{n\,m+2} - m \delta_{n\,m} - (m+1) \delta_{n\,m} \\
	 2 \braket{n|\xi^2|m} &= \sqrt{(m+1)(m+2)} \delta_{n\,m-2} + \sqrt{m(m-1)} \delta_{n\,m+2} + m \delta_{n\,m} + (m+1) \delta_{n\,m}
\end{align}

\begin{align}
	H_{n\,m} &= \braket{n|H|m}\\
			 &= m \delta_{n\,m} + (m+1)\delta_{n\,m}+\tilde{\lambda}\braket{n|\xi^4|m}	
\end{align}

Es ergaben sich folgende Eigenwerte.
\begin{table}[]
\centering
\caption{Matrixeigenwerte des Hamiltonians für $\tilde{\lambda}$ = 0.2}
\label{table:2blambeigenvalues}
\begin{tabular}{c}

\end{tabular}
\end{table}

Wir halten die Ergebnisse aus Aufgabenteil c für genauer, weil die Diskretisierung einen größeren Fehler als eine (entsprechend genügend große) Anzahl an Dimensionen $N$ des Unterraums generiert. Je größer der Unterraum, desto besser werden die gefundenen EW an die echten EW approximiert.

%\printbibliography
\end{document}
