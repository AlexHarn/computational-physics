\documentclass[a4paper, 11pt]{article}
\usepackage{longtable}
% LaTeX2e korrigieren.
\usepackage{fixltx2e}
% deutsche Spracheinstellungen
\usepackage{polyglossia}
\setmainlanguage{german}
\usepackage{diagbox}
\usepackage{fullpage}
% unverzichtbare Mathe-Befehle
\usepackage{amsmath}
% viele Mathe-Symbole
\usepackage{amssymb}
% Erweiterungen für amsmath
\usepackage{mathtools}

% Fonteinstellungen
\usepackage{fontspec}
\defaultfontfeatures{Ligatures=TeX}

\usepackage[
  math-style=ISO,    % \
  bold-style=ISO,    % |
  sans-style=italic, % | ISO-Standard folgen
  nabla=upright,     % |
  partial=upright,   % /
]{unicode-math}

\setmathfont{Latin Modern Math}
\setmathfont[range={\mathscr, \mathbfscr}]{XITS Math}
\setmathfont[range=\coloneq]{XITS Math}
\setmathfont[range=\propto]{XITS Math}
% make bar horizontal, use \hslash for slashed h
\let\hbar\relax
\DeclareMathSymbol{\hbar}{\mathord}{AMSb}{"7E}
\DeclareMathSymbol{ℏ}{\mathord}{AMSb}{"7E}

% richtige Anführungszeichen
\usepackage[autostyle]{csquotes}

% Zahlen und Einheiten
\usepackage[
  locale=DE,                   % deutsche Einstellungen
  separate-uncertainty=true,   % Immer Fehler mit \pm
  per-mode=symbol-or-fraction, % m/s im Text, sonst Brüche
]{siunitx}

% chemische Formeln
\usepackage[version=3]{mhchem}

% schöne Brüche im Text
\usepackage{xfrac}

% Floats innerhalb einer Section halten
\usepackage[section, below]{placeins}
% Captions schöner machen.
\usepackage[
  labelfont=bf,        % Tabelle x: Abbildung y: ist jetzt fett
  font=small,          % Schrift etwas kleiner als Dokument
  width=0.9\textwidth, % maximale Breite einer Caption schmaler
]{caption}
% subfigure, subtable, subref
\usepackage{subcaption}
%\usepackage{subfigure}

% Grafiken können eingebunden werden
\usepackage{graphicx}
% größere Variation von Dateinamen möglich
\usepackage{grffile}

% Standardplatzierung für Floats einstellen
\usepackage{float}
\floatplacement{figure}{htbp}
\floatplacement{table}{htbp}

% schöne Tabellen
\usepackage{booktabs}

% Seite drehen für breite Tabellen
\usepackage{pdflscape}

\usepackage{icomma}

% Mars und Venus
\usepackage{marvosym}

% Hyperlinks im Dokument
\usepackage[
  unicode,
  pdfusetitle,    % Titel, Autoren und Datum als PDF-Attribute
  pdfcreator={},  % PDF-Attribute säubern
  pdfproducer={}, % "
]{hyperref}
% erweiterte Bookmarks im PDF
\usepackage{bookmark}

% Trennung von Wörtern mit Strichen
\usepackage[shortcuts]{extdash}

\usepackage[ddmmyyyy]{datetime}
\renewcommand{\dateseparator}{.}
\usepackage[backend=biber]{biblatex}
\addbibresource../lit.bib}

\usepackage{verbatim}
\usepackage{ulem}
\usepackage{braket}
\newcommand\OverfullCenter[1]{\noindent\makebox[\linewidth]{#1}}
\begin{document}
\noindent
\centerline{\small{\textsc{Technische Universität Dortmund}}} \\
\large\textbf{Übungsblatt 10} \hfill \footnotesize\textbf{Sebastian Bange, Alexander Harnisch, Alexander Knodel} \\
\normalsize Computational Physics \hfill \today \\
Prof. Dr. Jan Kierfeld \hfill Abgabefrist: 01.07.2016\\
\noindent\makebox[\linewidth]{\rule{\textwidth}{0.4pt}}
\section*{Random-Walk in der NMR}
Die Störungen hängen der Einfachheit halber nicht vom Winkel $\varphi$ ab. In einer Kugel, bzw. Kugelkoordinaten sei $r$ ebenfalls konstant. Betrachtet man nun eine Kreisfläche (Kugelschnitte) mit variablen $r'$ (in der Horizontalen), ergibt sich folgendes.

\begin{align}
	l_{\text{Breitengrad}} &= 2\pi r' \\
	\sin(\theta) &= \frac{r'}{r} \\
	\rightarrow l_{\text{Breitengrad}} &= 2\pi r \sin(\theta) \\
	\rightarrow p(\theta) &= A \sin(\theta)\\
	\int_0^{\pi} p(\theta) \mathup{d\theta} &= \left[-A\cos(\theta)\right]_0^{\pi} = 2A \stackrel{!}{=} 1 \\
	\rightarrow p(\theta) &= \frac{1}{2} \sin(\theta)
\end{align}

\begin{landscape}
	\begin{figure}
		\OverfullCenter{\includegraphics[width=\textwidth]{../A1/a_hist.pdf}}
		\caption{Graphische Darstellung der Ergebnisse von Aufgabe 1a: Gleichverteilung der Funktion $p(\theta)$.}
		\label{fig:1a}
	\end{figure}
\end{landscape}
Das Legendre-Polynom zweiter Ordnung ($\delta = 1$) lautet
\begin{equation}
	\omega(\theta) = \frac{\delta}{2} (3\cos^2(\theta)-1) = \frac{1}{2} (3\cos^2(\theta)-1)
\end{equation}
Die numerisch bestimmte Verteilung ergibt sich zu
\begin{landscape}
	\begin{figure}
		\OverfullCenter{\includegraphics[width=\textwidth]{../A1/b_hist.pdf}}
		\caption{Graphische Darstellung der Ergebnisse von Aufgabe 1b: Gleichverteilung der Funktion $\omega(\theta)$.}
		\label{fig:1b}
	\end{figure}
\end{landscape}
Der Mittelwert von $\omega$ lautet: $\braket{\omega}$ = 0.000408532.
In Aufgabenteil c sollen nun die berechneten Erwartungswerte $M(t_p)$ angegeben werden.
\begin{landscape}
	\begin{figure}
		\OverfullCenter{\includegraphics[width=\textwidth]{../A1/c.pdf}}
		\caption{Graphische Darstellung der Ergebnisse von Aufgabe 1c: Erwartungswerte $M(t_p)$ gegen $t_p$.}
		\label{fig:1c}
	\end{figure}
\end{landscape}
\section*{Monte-Carlo Integration}
Der berechnete Wert von $\pi$ lautet: $\pi = 3.13157$.

\begin{landscape}
	\begin{figure}
		\OverfullCenter{\includegraphics[width=\textwidth]{../A2/b_hist.pdf}}
		\caption{Graphische Darstellung der Ergebnisse von Aufgabe 2b: Histogramm zur Verteilung von $\pi$.}
		\label{fig:2bhist}
	\end{figure}
\end{landscape}

\begin{landscape}
	\begin{figure}
		\OverfullCenter{\includegraphics[width=\textwidth]{../A2/b_errors.pdf}}
		\caption{Graphische Darstellung der Ergebnisse von Aufgabe 2b: Plot des Fehlers von $\pi$.}
		\label{fig:2bplot}
	\end{figure}
\end{landscape}

Für den Flächeninhalt der Ellipse erhalten wir als Plot.
\begin{landscape}
	\begin{figure}
		\OverfullCenter{\includegraphics[width=\textwidth]{../A2/c.pdf}}
		\caption{Graphische Darstellung der Ergebnisse von Aufgabe 2c: Zur Berechnung des Flächeninhalts der Ellipse.}
		\label{fig:1b}
	\end{figure}
\end{landscape}

Zum Schluss soll nun das Ergebnis des Integrals angegeben werden. Wir erhielten 2.993.
%TODO

%\printbibliography
\end{document}
