\documentclass[a4paper, 11pt]{article}
\usepackage{longtable}
% LaTeX2e korrigieren.
\usepackage{fixltx2e}
% deutsche Spracheinstellungen
\usepackage{polyglossia}
\setmainlanguage{german}
\usepackage{diagbox}
\usepackage{fullpage}
% unverzichtbare Mathe-Befehle
\usepackage{amsmath}
% viele Mathe-Symbole
\usepackage{amssymb}
% Erweiterungen für amsmath
\usepackage{mathtools}

% Fonteinstellungen
\usepackage{fontspec}
\defaultfontfeatures{Ligatures=TeX}

\usepackage[
  math-style=ISO,    % \
  bold-style=ISO,    % |
  sans-style=italic, % | ISO-Standard folgen
  nabla=upright,     % |
  partial=upright,   % /
]{unicode-math}

\setmathfont{Latin Modern Math}
\setmathfont[range={\mathscr, \mathbfscr}]{XITS Math}
\setmathfont[range=\coloneq]{XITS Math}
\setmathfont[range=\propto]{XITS Math}
% make bar horizontal, use \hslash for slashed h
\let\hbar\relax
\DeclareMathSymbol{\hbar}{\mathord}{AMSb}{"7E}
\DeclareMathSymbol{ℏ}{\mathord}{AMSb}{"7E}

% richtige Anführungszeichen
\usepackage[autostyle]{csquotes}

% Zahlen und Einheiten
\usepackage[
  locale=DE,                   % deutsche Einstellungen
  separate-uncertainty=true,   % Immer Fehler mit \pm
  per-mode=symbol-or-fraction, % m/s im Text, sonst Brüche
]{siunitx}

% chemische Formeln
\usepackage[version=3]{mhchem}

% schöne Brüche im Text
\usepackage{xfrac}

% Floats innerhalb einer Section halten
\usepackage[section, below]{placeins}
% Captions schöner machen.
\usepackage[
  labelfont=bf,        % Tabelle x: Abbildung y: ist jetzt fett
  font=small,          % Schrift etwas kleiner als Dokument
  width=0.9\textwidth, % maximale Breite einer Caption schmaler
]{caption}
% subfigure, subtable, subref
\usepackage{subcaption}
%\usepackage{subfigure}

% Grafiken können eingebunden werden
\usepackage{graphicx}
% größere Variation von Dateinamen möglich
\usepackage{grffile}

% Standardplatzierung für Floats einstellen
\usepackage{float}
\floatplacement{figure}{htbp}
\floatplacement{table}{htbp}

% schöne Tabellen
\usepackage{booktabs}

% Seite drehen für breite Tabellen
\usepackage{pdflscape}

\usepackage{icomma}

% Mars und Venus
\usepackage{marvosym}

% Hyperlinks im Dokument
\usepackage[
  unicode,
  pdfusetitle,    % Titel, Autoren und Datum als PDF-Attribute
  pdfcreator={},  % PDF-Attribute säubern
  pdfproducer={}, % "
]{hyperref}
% erweiterte Bookmarks im PDF
\usepackage{bookmark}

% Trennung von Wörtern mit Strichen
\usepackage[shortcuts]{extdash}

\usepackage[ddmmyyyy]{datetime}
\renewcommand{\dateseparator}{.}
\usepackage[backend=biber]{biblatex}
\addbibresource../lit.bib}

\usepackage{verbatim}
\usepackage{ulem}
\usepackage{braket}
\newcommand\OverfullCenter[1]{\noindent\makebox[\linewidth]{#1}}
\begin{document}
\noindent
\centerline{\small{\textsc{Technische Universität Dortmund}}} \\
\large\textbf{Übungsblatt 10} \hfill \footnotesize\textbf{Sebastian Bange, Alexander Harnisch, Alexander Knodel} \\
\normalsize Computational Physics \hfill \today \\
Prof. Dr. Jan Kierfeld \hfill Abgabefrist: 01.07.2016\\
\noindent\makebox[\linewidth]{\rule{\textwidth}{0.4pt}}
Im Allgemeinen sind die hier aufgeführten, numerisch bestimmten Werte Momentaufnahmen, da diese mittels Zufallszahlen bestimmt werden.
\section*{Random-Walk in der NMR}
Die Störungen hängen der Einfachheit halber nicht vom Winkel $\varphi$ ab. In einer Kugel, bzw. Kugelkoordinaten sei $r$ ebenfalls konstant. Betrachtet man nun eine Kreisfläche (Kugelschnitte) mit variablen $r'$ (in der Horizontalen), ergibt sich folgendes.

\begin{align}
	l_{\text{Breitengrad}} &= 2\pi r' \\
	\sin(\theta) &= \frac{r'}{r} \\
	\rightarrow l_{\text{Breitengrad}} &= 2\pi r \sin(\theta) \\
	\rightarrow p(\theta) &= A \sin(\theta)\\
	\int_0^{\pi} p(\theta) \mathup{d\theta} &= \left[-A\cos(\theta)\right]_0^{\pi} = 2A \stackrel{!}{=} 1 \\
	\rightarrow p(\theta) &= \frac{1}{2} \sin(\theta)
\end{align}

\begin{landscape}
	\begin{figure}
		\OverfullCenter{\includegraphics[width=\textwidth]{../A1/a_hist.pdf}}
		\caption{Graphische Darstellung der Ergebnisse von Aufgabe 1a: Gleichverteilung der Funktion $p(\theta)$.}
		\label{fig:1a}
	\end{figure}
\end{landscape}
Das Legendre-Polynom zweiter Ordnung ($\delta = 1$) lautet
\begin{equation}
	\omega(\theta) = \frac{\delta}{2} (3\cos^2(\theta)-1) = \frac{1}{2} (3\cos^2(\theta)-1)
\end{equation}
Die numerisch bestimmte Verteilung ergibt sich zu
\begin{landscape}
	\begin{figure}
		\OverfullCenter{\includegraphics[width=\textwidth]{../A1/b_hist.pdf}}
		\caption{Graphische Darstellung der Ergebnisse von Aufgabe 1b: Gleichverteilung der Funktion $\omega(\theta)$.}
		\label{fig:1b}
	\end{figure}
\end{landscape}
Der Mittelwert von $\omega$ lautet: $\braket{\omega}$ = 0.000154499. Theoretisch war hier der Mittelwert $\braket{\omega} = 0$ zu erwarten, was im Rahmen numerischen Rauschens erfüllt wurde.
In Aufgabenteil c sollen nun die berechneten Erwartungswerte $M(t_p)$ angegeben werden.
\begin{landscape}
	\begin{figure}
		\OverfullCenter{\includegraphics[width=\textwidth]{../A1/c.pdf}}
		\caption{Graphische Darstellung der Ergebnisse von Aufgabe 1c: Erwartungswerte $M(t_p)$ gegen $t_p$.}
		\label{fig:1c}
	\end{figure}
\end{landscape}
(vgl. Aufgabenblatt) Für sehr kleine Zeiten $t_p$ finden keine Sprünge statt, entsprechend ist zu erwarten,
dass für kleine $t_p$ die beiden Phasen gleich sind und daher der Erwarungswert der Magnetisierung 1 ist. Für zunehmendes $t_p$ finden Sprünge statt und werden die Korrelation zerstören, die Magnetisierung wird abnehmen und für sehr große $t_p$ gegen 0 gehen. Wie in Abb. \ref{fig:1c} erkennbar, beginnt die Magnetisierung nicht wie erwartet bei $M(t_p) = 1$, sondern etwas darunter und nimmt nicht den Verlauf einer gespiegelten Sigmoid-Kurve, sondern schwingt bei etwa $t_p \approx 10$ etwas unter 0, um dann in einer Art Rauschen gegen Null zu konvergieren.
\section*{Monte-Carlo Integration}
Der berechnete Wert von $\pi$ lautet: $\pi = 3.141$. Das Integral
\begin{equation}
\int_{|\vec{r}|<1} \mathup{d^2\,\textbf{r}}
\end{equation}
berechnet sich dadurch, dass nur Punkte innerhalb des Einheitskreises berücksichtigt werden, was zu 
\begin{equation}
\pi \approx \frac{4N_{\text{Kreis}}}{N}
\label{eq:piberechnung}
\end{equation} führt, mit $N_K$ als die Anzahl der Punkte im Einheitskreis und $N$ als die Gesamtanzahl der Punkte (alle Punkte liegen im Quadrat $[-1,1]^2$). 
Gleichung \ref{eq:piberechnung} ergibt sich aus der Wahrscheinlichkeit, die Punkte im Einheitskreis zu finden. Diese ist gegeben als das Verhältnis der Flächen von Einheitskreis und Quadrat.
\begin{equation}
p = \frac{A_{\text{Kreis}}}{A_{\text{Quadrat}}} = \frac{\pi}{4} \approx \frac{N_{\text{Kreis}}}{N}
\end{equation}
Anders formuliert: Das Verhältnis der Treffer im Einheitskreis zur Gesamtzahl der Versuche im Quadrat, verhält sich wie die entsprechenden Flächeninhalte bei im Quadrat gleichverteilten Versuchen.
Über die mittlere Zahl von Erfolgen ergibt sich:
\begin{equation}
\braket{N_{\text{kreis}}} = N\,p = N\frac{\pi}{4} \rightarrow \pi \approx \frac{4N_{\text{Kreis}}}{N}
\end{equation}
Mit $N_{\text{Kreis}} \approx \braket{N_{\text{Kreis}}}$ für große $N$ (da der Fehler $\sigma \propto 1/\sqrt{N}$).

Die Verteilung $p(x,y)$ entspricht einer Gleichverteilung in [0,1) (vgl. Aufgabe), wodurch sich die Wahrscheinlichkeit $p(x) = 1/2$ ergibt und die Funktion $f(x)$ (vgl. Gleichung \ref{eq:piberechnung}, Herleitung folgend) wird zu $f(x) = 4 \sqrt{1-x^2}$ (man beachte die Reduktion auf 1D-Abhängigkeit).
\begin{landscape}
	\begin{figure}
		\OverfullCenter{\includegraphics[width=\textwidth]{../A2/b_hist.pdf}}
		\caption{Graphische Darstellung der Ergebnisse von Aufgabe 2b: Histogramm zur Verteilung von $\pi$.}
		\label{fig:2bhist}
	\end{figure}
\end{landscape}

\begin{landscape}
	\begin{figure}
		\OverfullCenter{\includegraphics[width=\textwidth]{../A2/b_errors.pdf}}
		\caption{Graphische Darstellung der Ergebnisse von Aufgabe 2b: Plot des Fehlers von $\pi$.}
		\label{fig:2bplot}
	\end{figure}
\end{landscape}
Über den Fehler $\sigma$ wurde oben bereits geschrieben: $\sigma \propto 1/\sqrt{N}$. Entsprechend sollte die Verteilung gaußförmig aussehen.

Für den Flächeninhalt der Ellipse erhalten wir als Plot.
\begin{landscape}
	\begin{figure}
		\OverfullCenter{\includegraphics[width=\textwidth]{../A2/c.pdf}}
		\caption{Graphische Darstellung der Ergebnisse von Aufgabe 2c: Zur Berechnung des Flächeninhalts der Ellipse.}
		\label{fig:2c}
	\end{figure}
\end{landscape}

Zum Schluss soll nun das Ergebnis des Integrals angegeben werden. Wir erhielten \\$I$ = 2.99326.

%\printbibliography
\end{document}
