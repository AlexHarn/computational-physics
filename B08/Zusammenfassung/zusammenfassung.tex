\documentclass[a4paper, 11pt]{article}
\usepackage{longtable}
% LaTeX2e korrigieren.
\usepackage{fixltx2e}
% deutsche Spracheinstellungen
\usepackage{polyglossia}
\setmainlanguage{german}
\usepackage{diagbox}
\usepackage{fullpage}
% unverzichtbare Mathe-Befehle
\usepackage{amsmath}
% viele Mathe-Symbole
\usepackage{amssymb}
% Erweiterungen für amsmath
\usepackage{mathtools}

% Fonteinstellungen
\usepackage{fontspec}
\defaultfontfeatures{Ligatures=TeX}

\usepackage[
  math-style=ISO,    % \
  bold-style=ISO,    % |
  sans-style=italic, % | ISO-Standard folgen
  nabla=upright,     % |
  partial=upright,   % /
]{unicode-math}

\setmathfont{Latin Modern Math}
\setmathfont[range={\mathscr, \mathbfscr}]{XITS Math}
\setmathfont[range=\coloneq]{XITS Math}
\setmathfont[range=\propto]{XITS Math}
% make bar horizontal, use \hslash for slashed h
\let\hbar\relax
\DeclareMathSymbol{\hbar}{\mathord}{AMSb}{"7E}
\DeclareMathSymbol{ℏ}{\mathord}{AMSb}{"7E}

% richtige Anführungszeichen
\usepackage[autostyle]{csquotes}

% Zahlen und Einheiten
\usepackage[
  locale=DE,                   % deutsche Einstellungen
  separate-uncertainty=true,   % Immer Fehler mit \pm
  per-mode=symbol-or-fraction, % m/s im Text, sonst Brüche
]{siunitx}

% chemische Formeln
\usepackage[version=3]{mhchem}

% schöne Brüche im Text
\usepackage{xfrac}

% Floats innerhalb einer Section halten
\usepackage[section, below]{placeins}
% Captions schöner machen.
\usepackage[
  labelfont=bf,        % Tabelle x: Abbildung y: ist jetzt fett
  font=small,          % Schrift etwas kleiner als Dokument
  width=0.9\textwidth, % maximale Breite einer Caption schmaler
]{caption}
% subfigure, subtable, subref
\usepackage{subcaption}
%\usepackage{subfigure}

% Grafiken können eingebunden werden
\usepackage{graphicx}
% größere Variation von Dateinamen möglich
\usepackage{grffile}

% Standardplatzierung für Floats einstellen
\usepackage{float}
\floatplacement{figure}{htbp}
\floatplacement{table}{htbp}

% schöne Tabellen
\usepackage{booktabs}

% Seite drehen für breite Tabellen
\usepackage{pdflscape}

\usepackage{icomma}

% Mars und Venus
\usepackage{marvosym}

% Hyperlinks im Dokument
\usepackage[
  unicode,
  pdfusetitle,    % Titel, Autoren und Datum als PDF-Attribute
  pdfcreator={},  % PDF-Attribute säubern
  pdfproducer={}, % "
]{hyperref}
% erweiterte Bookmarks im PDF
\usepackage{bookmark}

% Trennung von Wörtern mit Strichen
\usepackage[shortcuts]{extdash}

\usepackage[ddmmyyyy]{datetime}
\renewcommand{\dateseparator}{.}
\usepackage[backend=biber]{biblatex}
\addbibresource../lit.bib}

\usepackage{verbatim}
\newcommand\OverfullCenter[1]{\noindent\makebox[\linewidth]{#1}}
\begin{document}
\noindent
\centerline{\small{\textsc{Technische Universität Dortmund}}} \\
\large\textbf{Übungsblatt 08} \hfill \footnotesize\textbf{Sebastian Bange, Alexander Harnisch, Alexander Knodel} \\
\normalsize Computational Physics \hfill \today \\
Prof. Dr. Jan Kierfeld \hfill Abgabefrist: 17.06.2016\\
\noindent\makebox[\linewidth]{\rule{\textwidth}{0.4pt}}
\section*{Bifurkationsdiagramme}
Die PDFs brauchen je nach Rechenleistung etwas Zeit zum Darstellen der Orbitpunkte.
Für die logistische Abbildung soll ein Bifurkationsdiagramm geplottet werden. Für 1a zeigt sich eine Aufspaltung bei $r \approx 3$, Chaos bei $r_{\infty} \approx 3.5$. $r_{\text{max}} = 4$. Darüber hinaus driftet das Diagramm über den gegebenen Definitionsbereich hinaus gen $-\infty$, bzw. $\infty$.
\begin{landscape}
	\begin{figure}
		\OverfullCenter{\includegraphics[width=\textwidth]{../A1/1a.pdf}}
		\caption{Graphische Darstellung der Ergebnisse von Aufgabe 1a: Plots des Bifurkationsdiagrammes}
		\label{fig:1a}
	\end{figure}
\end{landscape}
Bei 1b ergibt sich für die kubische Abbildung analoges Verhalten mit anderen Grenzen. Wir haben uns jeweils positive und negative Nullstellen angeschaut. Hier ergibt sich eine Aufspaltung bei $r \approx 2$, Chaos bei etwa $r_{\infty} \approx 2.35$, wo die Bifurkation versagt und auch positive $x$ sichtbar werden. $r_{\text{max}} = 3$, auch dort wird danach der Definitionsbereich überschritten (damit haben wir den Definitionsbereich für den Plot bestimmt). Wir haben eine Kalibrierungszeit von 1000 Schritten gewählt. Die Anzahl der $2^{\text{branches}}$ (und weitere Parameter der Simulation) werden dabei mitgegeben.
\begin{landscape}
	\begin{figure}
		\OverfullCenter{\includegraphics[width=\textwidth]{../A1/1b-pos.pdf}}
		\caption{Graphische Darstellung der Ergebnisse von Aufgabe 1b: Plots des Bifurkationsdiagrammes für positive Nullstellen}
		\label{fig:1bpos}
	\end{figure}
\end{landscape}
\begin{landscape}
	\begin{figure}
		\OverfullCenter{\includegraphics[width=\textwidth]{../A1/1b-neg.pdf}}
		\caption{Graphische Darstellung der Ergebnisse von Aufgabe 1b: Plots des Bifurkationsdiagrammes für negative Nullstellen}
		\label{fig:1bneg}
	\end{figure}
\end{landscape}

\section*{Feigenbaum-Konstante}
\begin{landscape}
	\begin{figure}
		\OverfullCenter{\includegraphics[width=\textwidth]{../A2/2a_n=0.pdf}}
		\caption{Graphische Darstellung der Ergebnisse von Aufgabe 2a: Plots von $g_n(r)$ (n=0,mit Wertebereich der Schranken)}
		\label{fig:2a_0_schranken}
	\end{figure}
\end{landscape}

\begin{landscape}
	\begin{figure}
		\OverfullCenter{\includegraphics[width=\textwidth]{../A2/2a_n=1.pdf}}
		\caption{Graphische Darstellung der Ergebnisse von Aufgabe 2a: Plots von $g_n(r)$ (n=1,gesamter Wertebereich)}
		\label{fig:2a_1}
	\end{figure}
\end{landscape}

\begin{landscape}
	\begin{figure}
		\OverfullCenter{\includegraphics[width=\textwidth]{../A2/2a_n=1_schranken.pdf}}
		\caption{Graphische Darstellung der Ergebnisse von Aufgabe 2a: Plots von $g_n(r)$ (n=1,Wertebereich der Schranken)}
		\label{fig:2a_1_schranken}
	\end{figure}
\end{landscape}

\begin{landscape}
	\begin{figure}
		\OverfullCenter{\includegraphics[width=\textwidth]{../A2/2a_n=2.pdf}}
		\caption{Graphische Darstellung der Ergebnisse von Aufgabe 2a: Plots von $g_n(r)$ (n=2,gesamter Wertebereich)}
		\label{fig:2a_2}
	\end{figure}
\end{landscape}

\begin{landscape}
	\begin{figure}
		\OverfullCenter{\includegraphics[width=\textwidth]{../A2/2a_n=2_schranken.pdf}}
		\caption{Graphische Darstellung der Ergebnisse von Aufgabe 2a: Plots von $g_n(r)$ (n=2,Wertebereich der Schranken)}
		\label{fig:2a_2_schranken}
	\end{figure}
\end{landscape}

\begin{landscape}
	\begin{figure}
		\OverfullCenter{\includegraphics[width=\textwidth]{../A2/2a_n=3.pdf}}
		\caption{Graphische Darstellung der Ergebnisse von Aufgabe 2a: Plots von $g_n(r)$ (n=3,gesamter Wertebereich)}
		\label{fig:2a_3}
	\end{figure}
\end{landscape}

\begin{landscape}
	\begin{figure}
		\OverfullCenter{\includegraphics[width=\textwidth]{../A2/2a_n=3_schranken.pdf}}
		\caption{Graphische Darstellung der Ergebnisse von Aufgabe 2a: Plots von $g_n(r)$ (n=3,Wertebereich der Schranken)}
		\label{fig:2a_3_schranken}
	\end{figure}
\end{landscape}
In Aufgabenteil b haben wir das Regula Falsi Verfahren gewählt.
\begin{enumerate}
\item $R_1$ = 3.173913
\item $R_2$ = 3.49856
\item $R_3$ = 3.554618
\end{enumerate}
In Aufgabenteil c erhalten wir die Feigenbaumkonstante 5.791291. In Aufgabenteil d erhalten wir
\begin{table}[]
	\centering
	\caption{$R_i$ für i $\geq$ 2, sowie die Feigenbaumkonstante $\delta_i$. }
	\label{RDelta}
	\begin{tabular}{cc}
		\toprule
		$R_i$ für i $\geq$ 2 & $\delta_i$   \\
		\midrule
		3.498562                     & 4.708943 \\
		3.554639                     & 4.68096   \\
		3.566667                     & 4.661894  \\
		3.569244                     & 4.669282   \\
		3.569795                     & 4.668954   \\
		3.569913                     & 4.669157   \\
		3.569939                     & 4.669191   \\
		3.569944                     & 4.6692     \\
		3.569945                     & 4.669201   \\
		3.569946                     & 4.669202   \\
		3.569946                     & 4.669201   \\
		3.569946                     & 4.669177   \\
		3.569946                     & 4.669337  \\
		\bottomrule
	\end{tabular}
\end{table}
Also $\delta = 4.669337$.
Es ist gut erkennbar, dass hier die Feigenbaum-Konstante genauer bestimmt wird, als
im vorherigen Aufgabenteil. Das war auch zu erwarten, da wir in Aufgabenteil b nur für $N$ = 3 iteriert
haben, während wir in Aufgabent il d $N$ = 15 wählen.

%\printbibliography
\end{document}
