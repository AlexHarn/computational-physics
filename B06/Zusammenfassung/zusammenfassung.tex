\documentclass[a4paper, 11pt]{article}
\usepackage{longtable}
% LaTeX2e korrigieren.
\usepackage{fixltx2e}
% deutsche Spracheinstellungen
\usepackage{polyglossia}
\setmainlanguage{german}
\usepackage{diagbox}
\usepackage{fullpage}
% unverzichtbare Mathe-Befehle
\usepackage{amsmath}
% viele Mathe-Symbole
\usepackage{amssymb}
% Erweiterungen für amsmath
\usepackage{mathtools}

% Fonteinstellungen
\usepackage{fontspec}
\defaultfontfeatures{Ligatures=TeX}

\usepackage[
  math-style=ISO,    % \
  bold-style=ISO,    % |
  sans-style=italic, % | ISO-Standard folgen
  nabla=upright,     % |
  partial=upright,   % /
]{unicode-math}

\setmathfont{Latin Modern Math}
\setmathfont[range={\mathscr, \mathbfscr}]{XITS Math}
\setmathfont[range=\coloneq]{XITS Math}
\setmathfont[range=\propto]{XITS Math}
% make bar horizontal, use \hslash for slashed h
\let\hbar\relax
\DeclareMathSymbol{\hbar}{\mathord}{AMSb}{"7E}
\DeclareMathSymbol{ℏ}{\mathord}{AMSb}{"7E}

% richtige Anführungszeichen
\usepackage[autostyle]{csquotes}

% Zahlen und Einheiten
\usepackage[
  locale=DE,                   % deutsche Einstellungen
  separate-uncertainty=true,   % Immer Fehler mit \pm
  per-mode=symbol-or-fraction, % m/s im Text, sonst Brüche
]{siunitx}

% chemische Formeln
\usepackage[version=3]{mhchem}

% schöne Brüche im Text
\usepackage{xfrac}

% Floats innerhalb einer Section halten
\usepackage[section, below]{placeins}
% Captions schöner machen.
\usepackage[
  labelfont=bf,        % Tabelle x: Abbildung y: ist jetzt fett
  font=small,          % Schrift etwas kleiner als Dokument
  width=0.9\textwidth, % maximale Breite einer Caption schmaler
]{caption}
% subfigure, subtable, subref
\usepackage{subcaption}
%\usepackage{subfigure}

% Grafiken können eingebunden werden
\usepackage{graphicx}
% größere Variation von Dateinamen möglich
\usepackage{grffile}

% Standardplatzierung für Floats einstellen
\usepackage{float}
\floatplacement{figure}{htbp}
\floatplacement{table}{htbp}

% schöne Tabellen
\usepackage{booktabs}

% Seite drehen für breite Tabellen
\usepackage{pdflscape}

\usepackage{icomma}

% Mars und Venus
\usepackage{marvosym}

% Hyperlinks im Dokument
\usepackage[
  unicode,
  pdfusetitle,    % Titel, Autoren und Datum als PDF-Attribute
  pdfcreator={},  % PDF-Attribute säubern
  pdfproducer={}, % "
]{hyperref}
% erweiterte Bookmarks im PDF
\usepackage{bookmark}

% Trennung von Wörtern mit Strichen
\usepackage[shortcuts]{extdash}

\usepackage[ddmmyyyy]{datetime}
\renewcommand{\dateseparator}{.}
\usepackage[backend=biber]{biblatex}
\addbibresource../lit.bib}

\usepackage{verbatim}
\newcommand\OverfullCenter[1]{\noindent\makebox[\linewidth]{#1}}
\begin{document}
\noindent
\centerline{\small{\textsc{Technische Universität Dortmund}}} \\
\large\textbf{Übungsblatt 06} \hfill \footnotesize\textbf{Sebastian Bange, Alexander Harnisch, Alexander Knodel} \\
\normalsize Computational Physics \hfill \today \\
Prof. Dr. Jan Kierfeld \hfill Abgabefrist: 03.06.2016\\
\noindent\makebox[\linewidth]{\rule{\textwidth}{0.4pt}}
\section*{Aufgabe 2 - Poisson-Gleichung}
Die zweidimensionale Poisson-Gleichung ($\varepsilon_0 = 1$)
\begin{equation}
\left( \partial_x^2 + \partial_y^2 \right)\,\phi(x,y) = -\rho(x,y)
\label{eq:Poisson-GL}
\end{equation}
wird mittels der Gauß-Seidel-Iteration nach (6.14) bei gegebener Genauigkeit $\epsilon$ gelöst. Zunächst sollte ein Quadrat $Q = [0,1]^2$, also Kantenlänge $L=1$, realisiert werden. Die Dirichlet Randbedingungen mit vorgegebenen Potential $\phi(x,y)$ auf den Rändern des Quadrates, sowie diskrete Ladungen $q_i$ im Inneren des Systems an den Orten $\vec{r_i}$ ($\rho(\vec{r}) = \sum_i\,q_i\delta (\vec{r}- \vec{r_i})$), wurden vorgegeben. Das System wurde über die Fixpunktgleichung mit $\Delta = 0.05$ diskretisiert.\\
\begin{enumerate}
\item[\textsc{calcP}] Wir berechnen das Potential mittels der Gauß-Seidel-Iteration, indem wir das Innere des Systems (also ohne Ränder) über die Fixpunktgleichung (6.18) bis zu einer Genauigkeit $\varepsilon = 10^{-5}$ iterieren. Dabei entspricht $i = j$ und $j = l$ (Programm $\Rightarrow$ Skript), also das Quadrat \enquote{von links nach rechts} rastern (und in jedem Schritt die y neu berechnen). Schließlich wird über den Absolutwert der Differenz zwischen aktueller und letztem Wert das jeweils aktuelle Maximum gesetzt.\\

\item[\textsc{calcE}] Über das Potential $\phi(x,y)$ wird das E-Feld berechnet:\\
\begin{equation}
\vec{E}(\vec{r}) = - \nabla\,\phi(x,y)
\end{equation}
Die Komponenten des Feldes werden über den Differenzenquotienten mit 2 symmetrischen Punkten (nach Formel (3.2)) berechnet.\\

\item[\textsc{PoissionRect}]
Wir initialisieren ein Quadrat der Maßen $[0,lx]\times[0,ly]$ bis zu einer Genauigkeit von $\varepsilon$. Dabei diskretisieren wir die $x$-Achse ($J$) und die $y$-Achse ($L$) über die Gitterkonstante $\Delta$.\\

\item[\textsc{setConstBC}]
Hier werden die RB an den vier Rändern auf die jeweils gegebene Konstante gesetzt. Die Reihenfolge ist eine Prioritätenliste, die Ecken werden entsprechend der Reihenfolge \enquote{links, rechts, unten, oben} überschrieben.\\

\item[\textsc{addQ}]
Fügt eine Ladung am Ort (x,y) mit dem Betrag $Q$ hinzu. Dabei wird erneut die Diskretisierung beachtet. Dies ist die Ladungsdichte $\rho(x,y)$.\\

\item[\textsc{save}]
Speichert die aktuellen Ergebnisse in Dateien des Formats \textsf{[name]\_[postfix].dat}. Beispielsweise werden so alle Potentialwerte $\phi(x,y)$, aber auch die Komponenten des E-Feldes gespeichert.\\

\item[\textsc{reset}]
Über eine reset-Methode kann zudem alles zentral zurückgesetzt werden.\\

\item[\textsc{main}]
Hier werden die einzelnen Methoden nur noch aufgerufen und mit den in den Teilaufgaben gegebenen Bedingungen ausgeführt.\\
\end{enumerate}

Es ergaben sich folgende Plots:

\subsection*{Aufgabenteil a}

\begin{landscape}
	\begin{figure}
		\OverfullCenter{\includegraphics[width=\textwidth]{../Abgabe/a.pdf}}
		\caption{Graphische Darstellung der Ergebnisse von Aufgabe 1a: $\phi(x,y)$}
		\label{fig:aPhi}
	\end{figure}
\end{landscape}

\begin{landscape}
	\begin{figure}
		\OverfullCenter{\includegraphics[width=\textwidth]{../Abgabe/a_absE.pdf}}
		\caption{Graphische Darstellung der Ergebnisse von Aufgabe 1a: $|E|(x,y)$}
		\label{fig:aabsE}
	\end{figure}
\end{landscape}


\subsection*{Aufgabenteil b}
In Aufgabenteil b mussten zusätzlich die Randbedingungen, bzw. praktisch nur die \enquote{neue} RB
$\phi(x,1)=1$ auf dem Rand $y=1$, berechnet werden.

\begin{landscape}
	\begin{figure}
		\OverfullCenter{\includegraphics[width=\textwidth]{../Abgabe/b.pdf}}
		\caption{Graphische Darstellung der Ergebnisse von Aufgabe 1b: $\phi(x,y)$\\links: Numerik
        (zusammen mit dem Feld), rechts: Analytik}
		\label{fig:bPhi}
	\end{figure}
\end{landscape}

\begin{landscape}
	\begin{figure}
		\OverfullCenter{\includegraphics[width=\textwidth]{../Abgabe/b_absE.pdf}}
		\caption{Graphische Darstellung der Ergebnisse von Aufgabe 1b: $|E|(x,y)$}
		\label{fig:babsE}
	\end{figure}
\end{landscape}
Die Lösung der Poisson-Gleichung \ref{eq:Poisson-GL} $\phi(x,y)$ wird über einen Separationsansatz gelöst.
\begin{align}
	\phi(x,y) &= X(x)\cdot Y(y)\\
	\left( \partial_x^2 + \partial_y^2 \right)\,\phi(x,y) &= 0 \\
	Y(y) \cdot \frac{\partial^2 X(x)}{\partial x^2} + X(x) \cdot \frac{\partial^2 Y(y)}{\partial y^2} &= 0 \qquad |\text{dividiere durch }\phi(x,y) = X(x)\cdot  Y(y)\\
	\frac{1}{X(x)} \cdot X'' + \frac{1}{Y(y)} \cdot Y'' &= 0 \\
	\frac{X''}{X} + \frac{Y''}{Y} &= 0 \\
	\frac{X''}{X} &=: \lambda^2 \\
    \Leftrightarrow X'' - \lambda^2 X &= 0 \qquad\text{(harmonischer Oszillator)} \\
    \Rightarrow X &= A\cos(\lambda x) + B\sin(\lambda y) \\
	\frac{Y''}{Y} &= -\lambda^2 \\
    \Leftrightarrow Y'' - (i\lambda)^2 Y &= 0 \qquad\text{(harmonischer Oszillator)} \\
    \Rightarrow Y &= C\cdot\cosh(\lambda y) + D\cdot\sinh(\lambda y) \\
    \Rightarrow \phi(x,y) &= \left(A\cos(\lambda x) + B\sin(\lambda y)\right)\cdot\left(C\cosh(\lambda y) + D\sinh(\lambda y)\right) \\
    \phi(0,y) &= A\cdot\left(C\cdot\cosh(\lambda y) + D\cdot\sinh(\lambda y)\right) = 0 \\
    \Rightarrow A &= 0 \\
    \phi(x,0) &= B\sin(\lambda x)\cdot C = 0 \\
    \Rightarrow C &= 0 \\
    \Rightarrow \phi(x,y) &= B\sin(\lambda x)\cdot D\sinh(\lambda y) \\
    &=: E\sin(\lambda x)\sinh(\lambda y) \\
    \phi(1,y) &= E\sin(\lambda)\sinh(\lambda y) = 0 \\
    \Rightarrow \lambda &= n\cdot\mathup{\pi}, \qquad n\in\mathbb{Z}
\end{align}
An dieser Stelle wird für die letzte Randbedingung nicht $\phi(x,1) = 1$ gewählt, da sonst die Punkte $(0,1)$ und $(1,1)$ beide gleichzeitig die Werte 1 und 0 haben müssten.
Außerdem müsste das Potential in diesen beiden Ecken springen, die Laplace-Gleichung liefert aber stetige Funktionen.
Ausweichend wird die Bedingung $\phi(0.5,1) = 1$ gewählt (obere Kante in der Mitte).
\begin{align}
    \phi(0.5,1) &= E\sin(0.5\cdot\lambda)\sinh(\lambda) = 1 \\
    \Rightarrow E &= \frac{1}{\sin(0.5\cdot\lambda)\sinh(\lambda)} \\
    \Rightarrow \lambda &= (2n+1)\cdot\mathup{\pi}, \qquad n\in\mathbb{Z} \\
    \Rightarrow \phi(x,y) &= \frac{\sin((2n+1)\cdot\mathup{\pi} x)\sinh((2n+1)\cdot\mathup{\pi} y)}{\sin(0.5\cdot(2n+1)\cdot\mathup{\pi})\sinh((2n+1)\cdot\mathup{\pi})}
\end{align}
Bei dem Vergleich von numerischer und analytischer Lösung (siehe Abbildung \ref{fig:bPhi}) fallen dabei direkt mehrere Dinge auf:
\begin{enumerate}
    \item Die obere Kante bei der analytischen Lösung ist nicht vollständig auf eins gesetzt.
    Das macht allerdings auch Sinn, da wir die Randbedingungen ja genau so abgeändert haben.
    \item Zum Plotten der analytischen Funktion wurde $n = 0$ gewählt, da alle Plots mit anders gewähltem $n$ deutlich von der numerischen Lösung abweichen.
    Auch dies macht Sinn, weil das Potential eigentlich an der kompletten oberen Kante den Wert 1 haben sollte.
    $n = 0$ sorgt für ein maximal homogenes Potential in x-Richtung, da nur an den beiden Rändern Knoten liegen.
    Bei einem (betragsmäßig) größeren $n$ wären noch Knoten an der oberen Kante entstanden.
    Dadurch würde das Potential dort jedoch noch weiter von eins abweichen.
    \item Abgesehen von der oberen Kante, ähneln sich die Potentiale stark, weswegen von einer grundsätzlich richtigen Implementation gesprochen werden kann.
\end{enumerate}
Unser Einwand zum Ändern der Randbedingung an der oberen Kante wird auch am Graphen in Abbildung \ref{fig:babsE} deutlich.
Die Unstetigkeit in den beiden Ecken sorgt für einen extrem großen Gradienten, weswegen das E-Feld an diesen Punkten deutlich größer ist als anderswo.
\subsection*{Aufgabenteil c}
Hier sollte nun eine Ladung $q_1 = +1$ in die Mitte von $Q$ ($x=0.5, y=0.5$) gesetzt werden. An den Rändern ist $\phi(x,y) = 0$, die Iteration soll bis $\varepsilon = 10^{-5}$ durchgeführt werden.

\begin{landscape}
	\begin{figure}
		\OverfullCenter{\includegraphics[width=\textwidth]{../Abgabe/c.pdf}}
		\caption{Graphische Darstellung der Ergebnisse von Aufgabe 1c: $\phi(x,y)$ und $-\nabla
        \phi(x,y)$}
		\label{fig:cPhi}
	\end{figure}
\end{landscape}

\begin{landscape}
	\begin{figure}
		\OverfullCenter{\includegraphics[width=\textwidth]{../Abgabe/c_absE.pdf}}
		\caption{Graphische Darstellung der Ergebnisse von Aufgabe 1c: $|E|(x,y)$}
		\label{fig:cabsE}
	\end{figure}
\end{landscape}

\subsection*{Aufgabenteil d}
\begin{align}		\int_{\partial Q} dl\,\sigma &= \int_{\partial Q} -\vec{n}\cdot \,\nabla \Phi = \int_{\partial Q} 	\vec{n}\cdot \vec{E}\,dl \stackrel{\text{Satz v. Gauß}}{=} \int_{Q} \text{div}\,\vec{E}\, dA \\
		&= \int_{Q} \left(\partial_x \,E_x+\partial_y\,E_y\right)\, dA
\end{align}%= \int_{0}^1 \int_{0}^1 \left(\partial_x \,E_x+\partial_y\,E_y\right)\, dx\,dy\\
Auf dem Rand des Quadrates wird das E-Feld mittels des einseitigen Differenzenquotienten berechnet.
Dazu wird komponentenweise $E_x$ (linker und rechter Rand), sowie $E_y$ (obiger und untiger Rand)
bestimmt. Die Gesamtinfluenz wird dabei über das Linienintegral (Ladungsdichte $\cdot$ Schrittweite)
über die Ladungsdichte über den Rand bestimmt. Im quadratischen System wird über alle 4 Seiten
summiert. Oben ($-E_y$) angefangen, wird im mathematisch positiven Sinn durchlaufen: $-E_y
\rightarrow E_x \rightarrow E_y \rightarrow -E_x$.
\begin{sloppypar}
Der numerisch erhaltene Wert für die influenzierte Oberflächenladung lautet \mbox{$q = -0.000890111$} und ist praktisch gleich dem theoretisch erwarteten Wert von Null.
Die doch unüblich große Abweichung von der Null folgt vermutlich aus der sehr groben Diskretisierung.
Der theoretische Wert von Null ergibt sich, weil die Ladung in der Mittel ein radialsymmetrisches Feld ausstrahlt.
Die Beträge von $E_\text{n}$ jeweils oben und unten, beziehungsweise links und rechts sind also gleich.
Da die Normalenvektoren immer in die gleiche Richtung wie $E_\text{n}$ stehen, sind die Integranden von jeweils oberer und unterer, beziehungsweise linker und rechter Wand gleich.
Es ergibt sich also:
\begin{equation}
    \sigma = \int_0^1 A dx + \int_0^1 B dy + \int_1^0 A dx + \int_1^0 B dy = 0.
\end{equation}
Ehrlich gesagt sind wir uns bei diesem Aufgabenteil ziemlich unsicher, vor allem weil die Ergebnisse
mit der schlechten Auflösung schwer zu überprüfen sind, viel bessere Auflösungen allerdings schon
viel zu hohe Laufzeiten benötigen.
\end{sloppypar}
\subsection*{Aufgabenteil e}
Nun sollte eine neutrale Ladungskonfiguration gewählt werden, mit mindestens zwei Ladungen (unter
der RB $\phi = 0$ auf allen vier Rändern). Dazu bietet sich der Dipol oder der Quadrupol an. Beides
haben wir implementiert. Für die speziellen von uns gewählten Verteilungen gilt wieder die selbe
Argumentation wie vorher. Die Potentiale bzw. Felder weisen eine Symmetrieachse auf, sodass das
Linienintegral verschwindet. Für asymmetrische Verteilungen gehen wir davon aus, dass überhaupt
keine analytische Lösung angegeben werden kann.

\subsubsection*{Dipol}
Als zweite Verteilung wählen wir einen Dipol mit Ladungen an den Punkten $q_1 = (0.25, 0.25)$ und
$q_2 = (0.75,0.75)$. Für die numerisch bestimmte influenzierte Ladung auf den Rändern erhalten wir
wie erwartet praktisch 0 ($10^{-18}$).

\begin{landscape}
	\begin{figure}
		\OverfullCenter{\includegraphics[width=\textwidth]{../Abgabe/e-dipol.pdf}}
		\caption{Graphische Darstellung der Ergebnisse von Aufgabe 1e: Dipol}
		\label{fig:edipol}
	\end{figure}
\end{landscape}


\begin{landscape}
	\begin{figure}
		\OverfullCenter{\includegraphics[width=\textwidth]{../Abgabe/e-dipol_absE.pdf}}
		\caption{Graphische Darstellung der Ergebnisse von Aufgabe 1e: Dipol: $|E|(x,y)$}
		\label{fig:edipolabse}
	\end{figure}
\end{landscape}

\subsubsection*{Quadrupol}
Als weitere Verteilung haben wir noch zwei zusätzlichen Ladungen $q_3 = (0.25,0.75)$ und $q_4 =
(0.75,0.25)$ zum Dipol hinzugefügt. Da diese Verteilung ebenfalls symmetrisch ist erhalten wir
wieder wie erwartet 0 für die induzierte Ladung.

\begin{landscape}
	\begin{figure}
		\OverfullCenter{\includegraphics[width=\textwidth]{../Abgabe/e-quadrupol.pdf}}
		\caption{Graphische Darstellung der Ergebnisse von Aufgabe 1e: Quadrupol}
		\label{fig:equad}
	\end{figure}
\end{landscape}

\begin{landscape}
	\begin{figure}
		\OverfullCenter{\includegraphics[width=\textwidth]{../Abgabe/e-quadrupol_absE.pdf}}
		\caption{Graphische Darstellung der Ergebnisse von Aufgabe 1e: Quadrupol: $|E|(x,y)$}
		\label{fig:equadabse}
	\end{figure}
\end{landscape}

%\printbibliography
\end{document}
