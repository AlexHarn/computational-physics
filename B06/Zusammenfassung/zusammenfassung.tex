\documentclass[a4paper, 11pt]{article}
\usepackage{longtable}
% LaTeX2e korrigieren.
\usepackage{fixltx2e}
% deutsche Spracheinstellungen
\usepackage{polyglossia}
\setmainlanguage{german}
\usepackage{diagbox}
\usepackage{fullpage}
% unverzichtbare Mathe-Befehle
\usepackage{amsmath}
% viele Mathe-Symbole
\usepackage{amssymb}
% Erweiterungen für amsmath
\usepackage{mathtools}

% Fonteinstellungen
\usepackage{fontspec}
\defaultfontfeatures{Ligatures=TeX}

\usepackage[
  math-style=ISO,    % \
  bold-style=ISO,    % |
  sans-style=italic, % | ISO-Standard folgen
  nabla=upright,     % |
  partial=upright,   % /
]{unicode-math}

\setmathfont{Latin Modern Math}
\setmathfont[range={\mathscr, \mathbfscr}]{XITS Math}
\setmathfont[range=\coloneq]{XITS Math}
\setmathfont[range=\propto]{XITS Math}
% make bar horizontal, use \hslash for slashed h
\let\hbar\relax
\DeclareMathSymbol{\hbar}{\mathord}{AMSb}{"7E}
\DeclareMathSymbol{ℏ}{\mathord}{AMSb}{"7E}

% richtige Anführungszeichen
\usepackage[autostyle]{csquotes}

% Zahlen und Einheiten
\usepackage[
  locale=DE,                   % deutsche Einstellungen
  separate-uncertainty=true,   % Immer Fehler mit \pm
  per-mode=symbol-or-fraction, % m/s im Text, sonst Brüche
]{siunitx}

% chemische Formeln
\usepackage[version=3]{mhchem}

% schöne Brüche im Text
\usepackage{xfrac}

% Floats innerhalb einer Section halten
\usepackage[section, below]{placeins}
% Captions schöner machen.
\usepackage[
  labelfont=bf,        % Tabelle x: Abbildung y: ist jetzt fett
  font=small,          % Schrift etwas kleiner als Dokument
  width=0.9\textwidth, % maximale Breite einer Caption schmaler
]{caption}
% subfigure, subtable, subref
\usepackage{subcaption}
%\usepackage{subfigure}

% Grafiken können eingebunden werden
\usepackage{graphicx}
% größere Variation von Dateinamen möglich
\usepackage{grffile}

% Standardplatzierung für Floats einstellen
\usepackage{float}
\floatplacement{figure}{htbp}
\floatplacement{table}{htbp}

% schöne Tabellen
\usepackage{booktabs}

% Seite drehen für breite Tabellen
\usepackage{pdflscape}

\usepackage{icomma}

% Mars und Venus
\usepackage{marvosym}

% Hyperlinks im Dokument
\usepackage[
  unicode,
  pdfusetitle,    % Titel, Autoren und Datum als PDF-Attribute
  pdfcreator={},  % PDF-Attribute säubern
  pdfproducer={}, % "
]{hyperref}
% erweiterte Bookmarks im PDF
\usepackage{bookmark}

% Trennung von Wörtern mit Strichen
\usepackage[shortcuts]{extdash}

\usepackage[ddmmyyyy]{datetime}
\renewcommand{\dateseparator}{.}
\usepackage[backend=biber]{biblatex}
\addbibresource../lit.bib}

\usepackage{verbatim}
\newcommand\OverfullCenter[1]{\noindent\makebox[\linewidth]{#1}}
\begin{document}
\noindent
\centerline{\small{\textsc{Technische Universität Dortmund}}} \\
\large\textbf{Übungsblatt 06} \hfill \footnotesize\textbf{Sebastian Bange, Alexander Harnisch, Alexander Knodel} \\
\normalsize Computational Physics \hfill \today \\
Prof. Dr. Jan Kierfeld \hfill Abgabefrist: 03.06.2016\\
\noindent\makebox[\linewidth]{\rule{\textwidth}{0.4pt}}
\section*{Aufgabe 2 - Poisson-Gleichung}
Die zweidimensionale Poisson-Gleichung ($\varepsilon_0 = 1$)
\begin{equation*}
\left( \partial_x^2 + \partial_y^2 \right)\,\phi(x,y) = -\rho(x,y)
\end{equation*}
wird mittels der Gauß-Seidel-Iteration nach (6.14) bei gegebener Genauigkeit $\epsilon$ gelöst. Zunächst sollte ein Quadrat $Q = [0,1]^2$, also Kantenlänge $L=1$, realisiert werden. Die Dirichlet Randbedingungen mit vorgegebenen Potential $\phi(x,y)$ auf den Rändern des Quadrates, sowie diskrete Ladungen $q_i$ im Inneren des Systems an den Orten $\vec{r_i}$ ($\rho(\vec{r}) = \sum_i\,q_i\delta (\vec{r}- \vec{r_i})$), wurden vorgegeben. Das System wurde über die Fixpunktgleichung mit $\Delta = 0.05$ diskretisiert.\\
\begin{enumerate}
\item[\textsc{calcP}] Wir berechnen das Potential mittels der Gauß-Seidel-Iteration, indem wir das Innere des Systems (also ohne Ränder) über die Fixpunktgleichung (6.18) bis zu einer Genauigkeit $\varepsilon = 10^{-5}$ iterieren. Dabei entspricht $i = j$ und $j = l$ (Programm $\Rightarrow$ Skript), also das Quadrat \enquote{von links nach rechts} rastern (und in jedem Schritt die y neu berechnen). Schließlich wird über den Absolutwert der Differenz zwischen aktueller und letztem Wert das jeweils aktuelle Maximum gesetzt.\\

\item[\textsc{calcE}] Über das Potential $\phi(x,y)$ wird das E-Feld berechnet:\\
\begin{equation*}
\vec{E}(\vec{r}) = - \nabla\,\phi(x,y)
\end{equation*}
Die Komponenten des Feldes werden über den Differenzenquotienten mit 2 symmetrischen Punkten (nach Formel (3.2)) berechnet.\\

\item[\textsc{PoissionRect}]
Wir initialisieren ein Quadrat der Maßen $[0,lx]\times[0,ly]$ bis zu einer Genauigkeit von $\varepsilon$. Dabei diskretisieren wir die $x$-Achse ($J$) und die $y$-Achse ($L$) über die Gitterkonstante $\Delta$.\\

\item[\textsc{setConstBC}]
Hier werden die RB an den vier Rändern auf die jeweils gegebene Konstante gesetzt. Die Reihenfolge ist eine Prioritätenliste, die Ecken werden entsprechend der Reihenfolge \enquote{links, rechts, unten, oben} überschrieben.\\

\item[\textsc{addQ}]
Fügt eine Ladung am Ort (x,y) mit dem Betrag $Q$ hinzu. Dabei wird erneut die Diskretisierung beachtet. Dies ist die Ladungsdichte $\rho(x,y)$.\\

\item[\textsc{save}]
Speichert die aktuellen Ergebnisse in Dateien des Formats \textsf{[name]\_[postfix].dat}. Beispielsweise werden so alle Potentialwerte $\phi(x,y)$, aber auch die Komponenten des E-Feldes gespeichert.\\

\item[\textsc{reset}]
Über eine reset-Methode kann zudem alles zentral zurückgesetzt werden.\\

\item[\textsc{main}]
Hier werden die einzelnen Methoden nur noch aufgerufen und mit den in den Teilaufgaben gegebenen Bedingungen ausgeführt.\\
\end{enumerate}

Es ergaben sich folgende Plots:

\subsection*{Aufgabenteil a}

\begin{landscape}
	\begin{figure}
		\OverfullCenter{\includegraphics[width=\textwidth]{../Abgabe/a.pdf}}
		\caption{Graphische Darstellung der Ergebnisse von Aufgabe 1a: $\phi(x,y)$}
		\label{fig:aPhi}
	\end{figure}
\end{landscape} 

\begin{landscape}
	\begin{figure}
		\OverfullCenter{\includegraphics[width=\textwidth]{../Abgabe/a_absE.pdf}}
		\caption{Graphische Darstellung der Ergebnisse von Aufgabe 1a: $|E|(x,y)$}
		\label{fig:aabsE}
	\end{figure}
\end{landscape} 


\subsection*{Aufgabenteil b}
In Aufgabenteil b mussten zusätzlich die Randbedingungen, bzw. praktisch nur die \enquote{neue} RB $\phi(x,1)=1$ auf dem Rand $y=1$, implementiert werden.

\begin{landscape}
	\begin{figure}
		\OverfullCenter{\includegraphics[width=\textwidth]{../Abgabe/b.pdf}}
		\caption{Graphische Darstellung der Ergebnisse von Aufgabe 1b: $\phi(x,y)$}
		\label{fig:bPhi}
	\end{figure}
\end{landscape} 

\begin{landscape}
	\begin{figure}
		\OverfullCenter{\includegraphics[width=\textwidth]{../Abgabe/b_absE.pdf}}
		\caption{Graphische Darstellung der Ergebnisse von Aufgabe 1b: $|E|(x,y)$}
		\label{fig:babsE}
	\end{figure}
\end{landscape} 

\subsection*{Aufgabenteil c}
Hier sollte nun eine Ladung $q_1 = +1$ in die Mitte von $Q$ ($x=0.5, y=0.5$) gesetzt werden. An den Rändern ist $\phi(x,y) = 0$, die Iteration soll bis $\varepsilon = 10^{-5}$ durchgeführt werden.

\begin{landscape}
	\begin{figure}
		\OverfullCenter{\includegraphics[width=\textwidth]{../Abgabe/c.pdf}}
		\caption{Graphische Darstellung der Ergebnisse von Aufgabe 1c: $\phi(x,y)$}
		\label{fig:cPhi}
	\end{figure}
\end{landscape} 

\begin{landscape}
	\begin{figure}
		\OverfullCenter{\includegraphics[width=\textwidth]{../Abgabe/c_absE.pdf}}
		\caption{Graphische Darstellung der Ergebnisse von Aufgabe 1c: $|E|(x,y)$}
		\label{fig:cabsE}
	\end{figure}
\end{landscape} 

\subsection*{Aufgabenteil e}
Nun sollte eine neutrale Ladungskonfiguration gewählt werden, mit mindestens zwei Ladungen (unter der RB $\phi = 0$ auf allen vier Rändern). Dazu bietet sich der Dipol oder der Quadrupol an. Beides haben wir implementiert.

\subsubsection*{Dipol}
Beim Dipol lässt sich eine neutrale Konfiguration über zwei entgegengerichtete Ladungen an den Punkten $q_1 = (0.25, 0.25)$ und $q_2 = (0.75,0.75)$ realisieren.

\begin{landscape}
	\begin{figure}
		\OverfullCenter{\includegraphics[width=\textwidth]{../Abgabe/e-dipol.pdf}}
		\caption{Graphische Darstellung der Ergebnisse von Aufgabe 1e: Dipol}
		\label{fig:edipol}
	\end{figure}
\end{landscape} 


\begin{landscape}
	\begin{figure}
		\OverfullCenter{\includegraphics[width=\textwidth]{../Abgabe/e-dipol_absE.pdf}}
		\caption{Graphische Darstellung der Ergebnisse von Aufgabe 1e: Dipol: $|E|(x,y)$}
		\label{fig:edipolabse}
	\end{figure}
\end{landscape} 

\subsubsection*{Quadrupol}
Analog zum Dipol lässt sich die neutrale Konfiguration über vier diskrete Ladungen realisieren, mit zwei zusätzlichen Punkten $q_3 = (0.25,0.75)$ und $q_4 = (0.75,0.25)$. Aus Neutralitätsgründen ändert sich das Vorzeichen bei Ladung $q_2$, während $q_3$ und $q_4$ negativ geladen sind.

\begin{landscape}
	\begin{figure}
		\OverfullCenter{\includegraphics[width=\textwidth]{../Abgabe/e-quadrupol.pdf}}
		\caption{Graphische Darstellung der Ergebnisse von Aufgabe 1e: Quadrupol}
		\label{fig:equad}
	\end{figure}
\end{landscape} 


\begin{landscape}
	\begin{figure}
		\OverfullCenter{\includegraphics[width=\textwidth]{../Abgabe/e-quadrupol_absE.pdf}}
		\caption{Graphische Darstellung der Ergebnisse von Aufgabe 1e: Quadrupol: $|E|(x,y)$}
		\label{fig:equadabse}
	\end{figure}
\end{landscape} 


%\printbibliography
\end{document}
